\documentclass{beamer}
\usetheme{Boadilla}
\usecolortheme[named=black]{structure}

\usepackage{listings}
\usepackage{color}
\usepackage{times}
\usepackage{tikz}
\usetikzlibrary{arrows, shapes}
\usetikzlibrary{positioning}

\title{Introduction to Compiler Construction}
\author{Fabian F\"ug}

\begin{document}

    %%https://tex.stackexchange.com/questions/6135/how-to-make-beamer-overlays-with-tikz-node
    \tikzset{onslide/.code args={<#1>#2}{%
    \only<#1>{\pgfkeysalso{#2}} % \pgfkeysalso doesn't change the path
    }}

    \begin{frame}
        \titlepage
    \end{frame}

    \section{Introduction}
    \begin{frame}
        \frametitle{Compiler}
        \tikzstyle{blackbox} = [rectangle, draw=black, text width = 3.37cm, minimum height=2.0cm, fill=orange]
        \begin{figure}
        \begin{tikzpicture}[node distance=0.7cm, auto, thick, >=latex']
            \node (box) [blackbox] {};
            
            \node (exe) [right=of box] {Executable};
            \node (text) [left=of box] {Text};

            \draw[->] (text) -- (box);
            \draw[->] (box) -- (exe);

        \end{tikzpicture}
        \end{figure}
    \end{frame}

    \begin{frame}
        \frametitle{Steps}

        \tikzstyle{stepsnode} = [rectangle, draw=black, text centered, text width=3.75cm, fill=orange]
        \begin{figure}
        \begin{tikzpicture}[node distance=0.7cm, auto, thick]

            \node (lexer) [stepsnode] {Lexer};
            \node (parser) [stepsnode, below=of lexer] {Parser};
            \node (sem) [stepsnode, below=of parser] {Semantic};
            \node (ir) [stepsnode, below=of sem] {IR Generation};
            \node (code) [stepsnode, below=of ir] {Code Generation};

            \node (exe) [right=of code] {Executable};
            \node (text) [left=of lexer] {Text};

            \draw[->] (lexer) edge node[swap] {Tokens} (parser);
            \draw[->] (parser) edge node[swap] {AST} (sem);
            \draw[->] (sem) edge node[swap] {AST} (ir);
            \draw[->] (ir) edge node[swap] {IR} (code);
            
            \draw[->] (text) -- (lexer);
            \draw[->] (code) -- (exe);

        \end{tikzpicture}
        \end{figure}
    \end{frame}

    \begin{frame}[fragile]
        \frametitle{Lexer}

        \tikzstyle{stepsnode} = [rectangle, draw=black, text centered, text width=3.75cm, fill=orange]
        \begin{figure}
        \begin{tikzpicture}[node distance=0.7cm, auto, thick]

            \node (lexer) [stepsnode] {Lexer};

            \node (tokens) [right=of lexer] {Tokens};
            \node (text) [left=of lexer] {Text};

            \draw[->] (text) -- (lexer);
            \draw[->] (lexer) -- (tokens);

        \end{tikzpicture}
        \end{figure}

        \begin{columns}
            \column{0.5\textwidth}
            \pause
            \begin{lstlisting}[language=c]
while(a < 10) {
    a += 1;
}
            \end{lstlisting}
            \column{0.5\textwidth}
            \pause
            \begin{figure}
            \tikzstyle{Token} = [rectangle, draw=black, text centered, fill=orange, minimum height=0.7cm]
            \begin{tikzpicture}[node distance=0.1cm, auto, thick]
    
                \node (a) [Token] {T\_While};
                \node (b) [Token, below=of a.south west, anchor=north west] {$($};
                \node (c) [Token, right=of b] {T\_Ident};
                \node (d) [Token, right=of c] {$<$};
                \node (e) [Token, right=of d] {T\_Int};
                \node (f) [Token, right=of e] {$)$};
                \node (g) [Token, below=of b] {$\{$};
                \node (h) [Token, below=of g.south west, anchor=north west] {T\_Ident};
                \node (i) [Token, right=of h] {$+=$};
                \node (j) [Token, right=of i] {T\_Int};
                \node (k) [Token, right=of j] {$;$};
                \node (l) [Token, below=of h.south west, anchor=north west] {$\}$};
    
            \end{tikzpicture}
            \end{figure}
        \end{columns}
    \end{frame}

    \begin{frame}
        \frametitle{Parser}

        \tikzstyle{stepsnode} = [rectangle, draw=black, text centered, text width=3.75cm, fill=orange]
        \begin{figure}
        \begin{tikzpicture}[node distance=0.7cm, auto, thick]

            \node (parser) [stepsnode] {Parser};

            \node (ast) [right=of parser] {AST};
            \node (tokens) [left=of parser] {Tokens};

            \draw[->] (tokens) -- (parser);
            \draw[->] (parser) -- (ast);

        \end{tikzpicture}
        \end{figure}

        \begin{columns}
            \column{0.5\textwidth}
            \pause
            \begin{figure}
            \tikzstyle{Token} = [rectangle, draw=black, text centered, fill=orange, minimum height=0.7cm]
            \begin{tikzpicture}[node distance=0.1cm, auto, thick]
    
                \node (a) [Token] {T\_While};
                \node (b) [Token, below=of a.south west, anchor=north west] {$($};
                \node (c) [Token, right=of b] {T\_Ident};
                \node (d) [Token, right=of c] {$<$};
                \node (e) [Token, right=of d] {T\_Int};
                \node (f) [Token, right=of e] {$)$};
                \node (g) [Token, below=of b] {$\{$};
                \node (h) [Token, below=of g.south west, anchor=north west] {T\_Ident};
                \node (i) [Token, right=of h] {$+=$};
                \node (j) [Token, right=of i] {T\_Int};
                \node (k) [Token, right=of j] {$;$};
                \node (l) [Token, below=of h.south west, anchor=north west] {$\}$};
    
            \end{tikzpicture}
            \end{figure}
            \column{0.5\textwidth}
            \pause
            \begin{figure}
            \tikzstyle{AST} = [rectangle, draw=black, text centered, fill=orange, minimum height=0.7cm]
            \begin{tikzpicture}[node distance=0.1cm, auto, thick]
    
                \node (a) [AST] {While};
                \node (b) [AST, below=of a, anchor=north west] {Binary};
                \node (c) [AST, right=of b] {$<$};
                \node (d) [AST, below=of b, anchor=north west] {Variable};
                \node (e) [AST, right=of d] {$a$};
                \node (f) [AST, below=of d.south west, anchor=north west] {Literal};
                \node (g) [AST, right=of f] {$10$};
                \node (h) [AST, below=2.6cm of a, anchor=north west] {Assign};
                \node (i) [AST, right=of h] {$...$};
    
            \end{tikzpicture}
            \end{figure}
        \end{columns}
    \end{frame}

    \begin{frame}
        \frametitle{Semantic}

        \tikzstyle{stepsnode} = [rectangle, draw=black, text centered, text width=3.75cm, fill=orange]
        \begin{figure}
        \begin{tikzpicture}[node distance=0.7cm, auto, thick]

            \node (sem) [stepsnode] {Semantic};

            \node (ast1) [right=of parser] {AST};
            \node (ast2) [left=of parser] {AST};

            \draw[->] (ast2) -- (sem);
            \draw[->] (sem) -- (ast1);

        \end{tikzpicture}
        \end{figure}

        \begin{columns}
            \column{0.5\textwidth}
            \pause
            \begin{figure}
            \tikzstyle{AST} = [rectangle, draw=black, text centered, fill=orange, minimum height=0.7cm]
            \begin{tikzpicture}[node distance=0.1cm, auto, thick]
    
                \node (a) [AST] {While};
                \node (b) [AST, below=of a, anchor=north west] {Binary};
                \node (c) [AST, right=of b] {$<$};
                \node (d) [AST, below=of b, anchor=north west] {Variable};
                \node (e) [AST, right=of d] {$a$};
                \node (f) [AST, below=of d.south west, anchor=north west] {Literal};
                \node (g) [AST, right=of f] {$10$};
                \node (h) [AST, below=2.6cm of a, anchor=north west] {Assign};
                \node (i) [AST, right=of h] {$...$};
    
            \end{tikzpicture}
            \end{figure}
        \column{0.5\textwidth}
        \pause
        \begin{figure}
        \tikzstyle{AST} = [rectangle, draw=black, text centered, fill=orange, minimum height=0.7cm]
        \begin{tikzpicture}[node distance=0.1cm, auto, thick]

            \node (a) [AST] {While};
            \node (b) [AST, below=of a, anchor=north west] {Binary};
            \node (c) [AST, right=of b] {$<$};
            \node (s1) [AST, right=of c] {Bool};
            \node (d) [AST, below=of b, anchor=north west] {Variable};
            \node (e) [AST, right=of d] {$a$};
            \node (s2) [AST, right=of e] {Int};
            \node (f) [AST, below=of d.south west, anchor=north west] {Literal};
            \node (g) [AST, right=of f] {$10$};
            \node (s3) [AST, right=of g] {Int};
            \node (h) [AST, below=2.6cm of a, anchor=north west] {Assign};
            \node (i) [AST, right=of h] {$...$};

        \end{tikzpicture}
        \end{figure}
        \end{columns}
    \end{frame}

    \begin{frame}[fragile]
        \frametitle{IR Generation}

        \tikzstyle{stepsnode} = [rectangle, draw=black, text centered, text width=3.75cm, fill=orange]
        \begin{figure}
        \begin{tikzpicture}[node distance=0.7cm, auto, thick]

            \node (irgen) [stepsnode] {IR Generation};

            \node (ir) [right=of parser] {IR};
            \node (ast) [left=of parser] {AST};

            \draw[->] (ast) -- (irgen);
            \draw[->] (irgen) -- (ir);

        \end{tikzpicture}
        \end{figure}

        \begin{columns}
            \column{0.5\textwidth}
            \pause
            \begin{figure}
            \tikzstyle{AST} = [rectangle, draw=black, text centered, fill=orange, minimum height=0.7cm]
            \begin{tikzpicture}[node distance=0.1cm, auto, thick]
    
                \node (a) [AST] {While};
                \node (b) [AST, below=of a, anchor=north west] {Binary};
                \node (c) [AST, right=of b] {$<$};
                \node (s1) [AST, right=of c] {Bool};
                \node (d) [AST, below=of b, anchor=north west] {Variable};
                \node (e) [AST, right=of d] {$a$};
                \node (s2) [AST, right=of e] {Int};
                \node (f) [AST, below=of d.south west, anchor=north west] {Literal};
                \node (g) [AST, right=of f] {$10$};
                \node (s3) [AST, right=of g] {Int};
                \node (h) [AST, below=2.6cm of a, anchor=north west] {Assign};
                \node (i) [AST, right=of h] {$...$};
    
            \end{tikzpicture}
            \end{figure}
            \column{0.5\textwidth}
            \pause
            \begin{lstlisting}[]
loop:
    gte a, 10, end
    add a, 1
    jmp loop
end:
            \end{lstlisting}
        \end{columns}
    \end{frame}

    \begin{frame}[fragile]
        \frametitle{Code Generation}

        \tikzstyle{stepsnode} = [rectangle, draw=black, text centered, text width=3.75cm, fill=orange]
        \begin{figure}
        \begin{tikzpicture}[node distance=0.7cm, auto, thick]

            \node (codegen) [stepsnode] {Code Generation};

            \node (code) [right=of parser] {Executable};
            \node (ir) [left=of parser] {IR};

            \draw[->] (ir) -- (codegen);
            \draw[->] (codegen) -- (code);

        \end{tikzpicture}
        \end{figure}

        \begin{columns}
            \column{0.5\textwidth}
            \pause
            \begin{lstlisting}[]
loop:
    gte a, 10, end
    add a, 1
    jmp loop
end:
            \end{lstlisting}
            \column{0.5\textwidth}
            \pause
            \begin{lstlisting}[]
.LBB0_1:
    cmp dword ptr [rbp - 8], 10
    jge .LBB0_3
    mov eax, dword ptr [rbp - 8]
    add eax, 1
    mov dword ptr [rbp - 8], eax
    jmp .LBB0_1
.LBB0_3:
            \end{lstlisting}
        \end{columns}
    \end{frame}

    \begin{frame}
        \frametitle{Frontend/Backend}

        \tikzstyle{stepsnode} = [rectangle, draw=black, text centered, text width=3.75cm, fill=orange]
        \tikzstyle{frontend} = [stepsnode, fill=orange]
        \tikzstyle{backend} = [stepsnode, fill=red]
        \begin{figure}
        \begin{tikzpicture}[node distance=0.7cm, auto, thick]

            \node (lexer) [frontend] {Lexer};
            \node (parser) [frontend, below=of lexer] {Parser};
            \node (sem) [frontend, below=of parser] {Semantic};
            \node (ir) [frontend, below=of sem] {IR Generation};
            \node (code) [backend, below=of ir] {Code Generation};

            \node (exe) [right=of code] {Executable};
            \node (text) [left=of lexer] {Text};

            \draw[->] (lexer) edge node[swap] {Tokens} (parser);
            \draw[->] (parser) edge node[swap] {AST} (sem);
            \draw[->] (sem) edge node[swap] {AST} (ir);
            \draw[->] (ir) edge node[swap] {IR} (code);
            
            \draw[->] (text) -- (lexer);
            \draw[->] (code) -- (exe);

        \end{tikzpicture}
        \end{figure}
    \end{frame}

    \begin{frame}
        \frametitle{Interpreter}

        \tikzstyle{stepsnode} = [rectangle, draw=black, text centered, text width=3.75cm, fill=orange]
        \tikzstyle{frontend} = [stepsnode, fill=orange]
        \tikzstyle{backend} = [stepsnode, fill=red]
        \begin{figure}
        \begin{tikzpicture}[node distance=0.7cm, auto, thick]

            \node (lexer) [frontend] {Lexer};
            \node (parser) [frontend, below=of lexer] {Parser};
            \node (ir) [frontend, below=of parser] {Interpreter};

            \node (text) [left=of lexer] {Text};

            \draw[->] (lexer) edge node[swap] {Tokens} (parser);
            \draw[->] (parser) edge node[swap] {AST} (ir);
            
            \draw[->] (text) -- (lexer);

        \end{tikzpicture}
        \end{figure}
    \end{frame}

    \definecolor{deepred}{rgb}{0.7,0,0}
    \definecolor{deepgreen}{rgb}{0,0.5,0}

    \lstset{
        language=Python,
        keywordstyle=\color{deepred},
        otherkeywords={self, let},
        stringstyle=\color{deepgreen},
        frame=tb,
        xleftmargin=.2\textwidth, xrightmargin=.2\textwidth
    }
    
    \begin{frame}[fragile]
        \frametitle{CPU lang}
        \begin{lstlisting}[]
let a = 0;
let b = 1;
let counter = 0;
let fib = 10;
while(counter < fib) {
    let tmp = b;
    b = a + b;
    a = tmp;
    counter = counter + 1;
}
print fib + ". fibonacci number: " + b;
        \end{lstlisting}
    \end{frame}

    \begin{frame}
        \frametitle{DEMO}
        \centering
        \textbf{https://github.com/Islidius/compiler-construction-presentation}
    \end{frame}

    \begin{frame}[fragile]
        \frametitle{Lexer}

        \tikzstyle{char} = [rectangle, draw=black, text centered, text width=0.4cm, minimum height=0.7cm]
        \tikzstyle{token} = [rectangle, draw=black, text centered, minimum height=0.7cm, fill=orange, opacity=0]
        \tikzstyle{high} = [fill=orange]
        \begin{figure}
        \begin{tikzpicture}[node distance=0.1cm, auto, thick]

            \node (a1) [char, onslide=<2->{high}] {w};
            \node (a2) [char, onslide=<3->{high}, right=of a1] {h};
            \node (a3) [char, onslide=<4->{high}, right=of a2] {i};
            \node (a4) [char, onslide=<5->{high}, right=of a3] {l};
            \node (a5) [char, onslide=<6->{high}, right=of a4] {e};
            \node (a6) [char, right=of a5] {~};

            \node (a7) [char, right=of a6] {$($};
            \node (a8) [char, right=of a7] {a};
            \node (a9) [char, right=of a8] {$<$};
            \node (a10) [char, right=of a9] {$1$};
            \node (a11) [char, right=of a10] {$0$};
            \node (a12) [char, right=of a11] {$)$};

            \node (while) [token, , onslide=<7->{opacity=1}, text width=3.55cm, below=of a1.south west, anchor=north west] {while};

        \end{tikzpicture}
        \end{figure}
    \end{frame}

    \begin{frame}[fragile]
        \frametitle{Maximal Munch}
        
        \tikzstyle{char} = [rectangle, draw=black, text centered, text width=0.4cm, minimum height=0.7cm]
        \tikzstyle{char2} = [rectangle, draw=black, text centered, text width=0.4cm, minimum height=0.7cm, opacity=0, onslide=<2->{opacity=1}]
        \tikzstyle{token} = [rectangle, draw=black, text centered, minimum height=0.7cm, fill=orange]
        \tikzstyle{high} = [fill=orange]
        \begin{figure}
        \begin{tikzpicture}[node distance=0.1cm, auto, thick]

            \node (a1) [char, high] {f};
            \node (a2) [char, high, right=of a1] {o};
            \node (a3) [char, high, right=of a2] {r};
            \node (a4) [char, right=of a3] {~};
            \node (a5) [char, right=of a4] {~};
            \node (a6) [char, right=of a5] {~};
            \node (a7) [char, right=of a6] {~};
            \node (a8) [char, right=of a7] {~};

            \node (for) [token, text width=2.0cm, below=of a1.south west, anchor=north west] {for};

            \node (b1) [char2, onslide=<3->{high}, below=of for.south west, anchor=north west] {f};
            \node (b2) [char2, onslide=<4->{high}, right=of b1] {o};
            \node (b3) [char2, onslide=<5->{high}, right=of b2] {r};
            \node (b4) [char2, right=of b3] {t};
            \node (b5) [char2, right=of b4] {r};
            \node (b6) [char2, right=of b5] {e};
            \node (b7) [char2, right=of b6] {s};
            \node (b8) [char2, right=of b7] {s};

            \node (for) [token, opacity=0, onslide=<6->{opacity=1}, text width=2.0cm, below=of b1.south west, anchor=north west] {for};

        \end{tikzpicture}
        \end{figure}
    \end{frame}

    \begin{frame}[fragile]
        \frametitle{Problems}
        
        \tikzstyle{char} = [rectangle, draw=black, text centered, text width=0.4cm, minimum height=0.7cm]
        \tikzstyle{token} = [rectangle, draw=black, text centered, text width=0.4cm, minimum height=0.7cm, fill=orange]
        \begin{figure}
        \begin{tikzpicture}[node distance=0.1cm, auto, thick]

            \node (a1) [char] {A};
            \node (a2) [char, right=of a1] {$<$};
            \node (a3) [char, right=of a2] {B};
            \node (a4) [char, right=of a3] {$<$};
            \node (a5) [char, right=of a4] {S};
            \node (a6) [char, right=of a5] {$>$};
            \node (a7) [char, right=of a6] {~};
            \node (a8) [char, right=of a7] {$>$};

            \node (c1) [token, below=of a1.south west, anchor=north west] {i};
            \node (c2) [token, right=of c1] {$<$};
            \node (c3) [token, right=of c2] {i};
            \node (c4) [token, right=of c3] {$<$};
            \node (c5) [token, right=of c4] {i};
            \node (c6) [token, right=of c5] {$>$};
            \node (c8) [token, below=of a8] {$>$};

            \node (b1) [char, below=of c1.south west, anchor=north west] {A};
            \node (b2) [char, right=of b1] {$<$};
            \node (b3) [char, right=of b2] {B};
            \node (b4) [char, right=of b3] {$<$};
            \node (b5) [char, right=of b4] {S};
            \node (b6) [char, right=of b5] {$>$};
            \node (b7) [char, right=of b6] {$>$};
            \node (b8) [char, right=of b7] {~};

            \node (d1) [token, below=of b1.south west, anchor=north west] {i};
            \node (d2) [token, right=of d1] {$<$};
            \node (d3) [token, right=of d2] {i};
            \node (d4) [token, right=of d3] {$<$};
            \node (d5) [token, right=of d4] {i};
            \node (d6) [token, right=of d5, text width=1.2cm] {$>>$};

        \end{tikzpicture}
        \end{figure}
    \end{frame}

    \begin{frame}[fragile]
        \frametitle{CPU Tokens}
        \begin{columns}
        \column{0.5\textwidth}
        \begin{lstlisting}[]
let a = 0;
let b = 1;
let counter = 0;
let fib = 10;
while(counter < fib) {
    let tmp = b;
    b = a + b;
    a = tmp;
    counter = counter + 1;
}
print "fibonacci: " + b;
        \end{lstlisting}
        \column{0.5\textwidth}
        \begin{figure}
            \tikzstyle{Token} = [rectangle, draw=black, text centered, fill=orange, minimum height=0.7cm, opacity=0, onslide=<2->{opacity=1}]
            \tikzstyle{Token2} = [rectangle, draw=black, text centered, fill=orange, minimum height=0.7cm, opacity=0, onslide=<3->{opacity=1}]
            \begin{tikzpicture}[node distance=0.1cm, auto, thick]
    
                \node (a) [Token] {T\_While};
                \node (b) [Token, right=of a] {T\_If};

                \node (c) [Token, below=of a.south west, anchor=north west] {T\_Let};
                \node (d) [Token, right=of c] {T\_Print};

                \node (e) [Token, below=of c.south west, anchor=north west] {T\_Ident};

                \node (f) [Token, below=of e.south west, anchor=north west] {T\_Number};
                \node (g) [Token, right=of f] {T\_String};

                \node (h) [Token2, below=of f.south west, anchor=north west] {$($};
                \node (i) [Token2, right=of h] {$)$};
                \node (j) [Token2, right=of i] {$\{$};
                \node (k) [Token2, right=of j] {$\}$};
                \node (l) [Token2, right=of k] {$=$};
                \node (m) [Token2, right=of l] {$;$};
                \node (n) [Token2, right=of m] {$"$};

                \node (o) [Token2, below=of h.south west, anchor=north west] {$+$};
                \node (p) [Token2, right=of o] {$-$};
                \node (q) [Token2, right=of p] {$*$};
                \node (r) [Token2, right=of q] {$/$};

                \node (s) [Token2, below=of o.south west, anchor=north west] {$==$};
                \node (t) [Token2, right=of s] {$<$};
                \node (u) [Token2, right=of t] {$>$};
                \node (v) [Token2, right=of u] {$<=$};
            \end{tikzpicture}
            \end{figure}
        \end{columns}
    \end{frame}

    \begin{frame}
        \frametitle{Handrolled Lexer}
        \tikzstyle{char} = [rectangle, draw=black, text centered, text width=0.4cm, minimum height=0.7cm]
        \tikzstyle{token} = [rectangle, draw=black, text centered, minimum height=0.7cm, fill=orange, opacity=0]
        \tikzstyle{high} = [fill=orange]
        \begin{figure}
        \begin{tikzpicture}[node distance=0.1cm, auto, thick]

            \node (a1) [char, onslide=<2->{high}] {w};
            \node (a2) [char, onslide=<3->{high}, right=of a1] {h};
            \node (a3) [char, onslide=<4->{high}, right=of a2] {i};
            \node (a4) [char, onslide=<5->{high}, right=of a3] {l};
            \node (a5) [char, onslide=<6->{high}, right=of a4] {e};
            \node (a6) [char, onslide=<7>{high}, right=of a5] {~};

            \node (a7) [char, right=of a6] {$($};
            \node (a8) [char, right=of a7] {a};
            \node (a9) [char, right=of a8] {$<$};
            \node (a10) [char, right=of a9] {$1$};
            \node (a11) [char, right=of a10] {$0$};
            \node (a12) [char, right=of a11] {$)$};

            \node (while) [token, onslide=<8->{opacity=1}, text width=3.55cm, below=of a1.south west, anchor=north west] {while};

        \end{tikzpicture}
        \end{figure}
    \end{frame}

    \begin{frame}[fragile]
        \frametitle{Handrolled Lexer}
        \begin{figure}
            \begin{lstlisting}[language=Python]
class Kind(Enum):
    NUMBER = 1
    STRING = 2
    TRUE = 3
    FALSE = 4
    ...
            \end{lstlisting}
            \centering
        \end{figure}
    \end{frame}

    \begin{frame}[fragile]
        \begin{figure}
        \centering
        \begin{lstlisting}[language=Python]
class Token(object):
    def __init__(self, kind, line, value):
        self.kind = kind
        self.value = value
        self.line = line
        \end{lstlisting}
        \end{figure}
    \end{frame}

    
    \begin{frame}[fragile]
        \begin{lstlisting}[language=Python]
def lexTokens(self):
    while(self.has()):
        self.start = self.index
        self.lexToken()
    
    self.addToken(Kind.EOF)
    return self.tokens
        \end{lstlisting}
    \end{frame}

    \begin{frame}[fragile]
    \begin{lstlisting}[language=Python]
def lexToken(self):
    char = self.next()
    if(char in " \r\t"):
        pass
    elif(char == "\n"):
        self.line += 1
    \end{lstlisting}
    \end{frame}

    \begin{frame}[fragile]
    \begin{lstlisting}[language=Python]
    elif(char in self.digits):
        self.lexNumber()
    \end{lstlisting}
    \begin{lstlisting}[language=Python]
def lexNumber(self):
    while(self.peek() in self.digits):
        self.next()

    self.addToken(
        Kind.NUMBER,
        int(self.getRange())
    )
    \end{lstlisting}
    \end{frame}

    \begin{frame}[fragile]
    \begin{lstlisting}[language=Python]
elif(char in self.alphas):
    self.lexIdentifier()
    \end{lstlisting}
    \begin{lstlisting}[language=Python]
def lexIdentifier(self):
    while(self.peek() in self.digits
        or self.peek() in self.alphas):
        self.next()

    value = self.getRange()

    if(value in self.keywords):
        self.addToken(self.keywords[value])
    else:
        self.addToken(Kind.IDENT, value)
    \end{lstlisting}
    \end{frame}

    \begin{frame}[fragile]
    \begin{lstlisting}[language=Python]
elif(char == "+"):
    self.addToken(Kind.PLUS)
    \end{lstlisting}
    \begin{lstlisting}[language=Python]
elif(char == "="):
    if(self.peek() == "="):
        self.next()
        self.addToken(Kind.CMPEQ)
    else:
        self.addToken(Kind.EQUALS)
    \end{lstlisting}
    \end{frame}

    \begin{frame}
        \frametitle{Parser}
        \begin{figure}
        \tikzstyle{Token} = [rectangle, draw=black, text centered, fill=orange, text width=0.3cm, minimum height=0.7cm]
        \tikzstyle{TokenInt} = [rectangle, draw=black, text centered, fill=orange, text width=1.3cm, minimum height=0.7cm]
        \begin{tikzpicture}[node distance=0.1cm, auto, thick]
            \node (a1) [TokenInt] {int};
            \node (a2) [Token, right=of a1] {$+$};
            \node (a3) [Token, right=of a2] {$($};
            \node (a4) [TokenInt, right=of a3] {int};
            \node (a5) [Token, right=of a4] {$+$};
            \node (a6) [TokenInt, right=of a5] {int};
            \node (a7) [Token, right=of a6] {$)$};
        \end{tikzpicture}
        \end{figure}
    \end{frame}

    \begin{frame}[fragile]
        \frametitle{Grammar}
        \begin{verbatim}
            S = S "+" S
            S = "(" S ")"
            S = "int"
        \end{verbatim}

        \begin{verbatim}
            N = D | N D
            D = "0" | "1" | "2" | "3"

            N = D D*
        \end{verbatim}
    \end{frame}

    \begin{frame}[fragile]
        \frametitle{Grammar}
        \begin{figure}
        \tikzstyle{Token} = [rectangle, draw=black, text centered, fill=orange, text width=0.3cm, minimum height=0.7cm]
        \tikzstyle{TokenInt} = [rectangle, draw=black, text centered, fill=orange, text width=1.3cm, minimum height=0.7cm]
        \begin{tikzpicture}[node distance=0.1cm, auto, thick]
            \node (a1) [TokenInt] {int};
            \node (a2) [Token, right=of a1] {$+$};
            \node (a3) [Token, right=of a2] {$($};
            \node (a4) [TokenInt, right=of a3] {int};
            \node (a5) [Token, right=of a4] {$+$};
            \node (a6) [TokenInt, right=of a5] {int};
            \node (a7) [Token, right=of a6] {$)$};
        \end{tikzpicture}
        \end{figure}
        \begin{columns}
        \column{0.5\textwidth}
        \begin{verbatim}
    S = S "+" S
    S = "(" S ")"
    S = "int"
        \end{verbatim}
        \column{0.5\textwidth}
        \begin{figure}
        \tikzstyle{Token} = [rectangle, draw=black, text centered, fill=orange, text width=1.0cm, minimum height=0.7cm, opacity=0]
        \tikzstyle{Show} = [opacity=1]
        \begin{tikzpicture}[node distance=0.2cm, auto, thick]
            \node (a1) [Token, onslide=<2->{Show}] {S};
            \node (a2) [Token, onslide=<3->{Show}, below=of a1] {S + S};
            \node (a3) [Token, onslide=<4->{Show}, below left=0.2cm and 0.1cm of a2.south west, anchor=north] {int};
            \node (a4) [Token, onslide=<5->{Show}, below right=0.2cm and 0.1cm of a2.south east, anchor=north] {$($S$)$};
            \node (a5) [Token, onslide=<6->{Show}, below=of a4] {S + S};
            \node (a6) [Token, onslide=<7->{Show}, below left=0.2cm and 0.1cm of a5.south west, anchor=north] {int};
            \node (a7) [Token, onslide=<8->{Show}, below right=0.2cm and 0.1cm of a5.south east, anchor=north] {int};

            \draw<3-> [->] (a1) -- (a2);
            \draw<4-> [->] (a2) -- (a3);
            \draw<5-> [->] (a2) -- (a4);
            \draw<6-> [->] (a4) -- (a5);
            \draw<7-> [->] (a5) -- (a6);
            \draw<8-> [->] (a5) -- (a7);

        \end{tikzpicture}
        \end{figure}
        \end{columns}
    \end{frame}

    \begin{frame}[fragile]
        \frametitle{Grammar - Backtracking?}
        \begin{figure}
        \tikzstyle{Token} = [rectangle, draw=black, text centered, fill=orange, text width=0.3cm, minimum height=0.7cm]
        \tikzstyle{TokenInt} = [rectangle, draw=black, text centered, fill=orange, text width=1.3cm, minimum height=0.7cm]
        \begin{tikzpicture}[node distance=0.1cm, auto, thick]
            \node (a1) [TokenInt] {int};
            \node (a2) [Token, right=of a1] {$+$};
            \node (a3) [Token, right=of a2] {$($};
            \node (a4) [TokenInt, right=of a3] {int};
            \node (a5) [Token, right=of a4] {$+$};
            \node (a6) [TokenInt, right=of a5] {int};
            \node (a7) [Token, right=of a6] {$)$};
        \end{tikzpicture}
        \end{figure}
        \begin{columns}
        \column{0.5\textwidth}
        \begin{verbatim}
    S = S "+" S
    S = "(" S ")"
    S = "int"
        \end{verbatim}
        \column{0.5\textwidth}
        \begin{figure}
        \tikzstyle{Token} = [rectangle, draw=black, text centered, fill=orange, text width=1.0cm, minimum height=0.7cm]
        \begin{tikzpicture}[node distance=0.2cm, auto, thick]
            \node (a1) [Token] {S};
            \node (a2) [Token, below=of a1] {S + S};
            \node (a3) [Token, below left=0.2cm and 0.1cm of a2.south west, anchor=north] {S + S};

            \draw[->] (a1) -- (a2);
            \draw[->] (a2) -- (a3);
        \end{tikzpicture}
        \end{figure}
        \end{columns}
    \end{frame}

    \begin{frame}[fragile]
        \frametitle{Grammar - Lookahead}
        \begin{figure}
        \tikzstyle{Token} = [rectangle, draw=black, text centered, fill=orange, text width=0.3cm, minimum height=0.7cm]
        \tikzstyle{TokenInt} = [rectangle, draw=black, text centered, fill=orange, text width=1.3cm, minimum height=0.7cm]
        \begin{tikzpicture}[node distance=0.1cm, auto, thick]
            \node (a1) [TokenInt, onslide=<2->{fill=red}] {int};
            \node (a2) [Token, right=of a1] {$+$};
            \node (a3) [Token, right=of a2] {$($};
            \node (a4) [TokenInt, right=of a3] {int};
            \node (a5) [Token, right=of a4] {$+$};
            \node (a6) [TokenInt, right=of a5] {int};
            \node (a7) [Token, right=of a6] {$)$};
        \end{tikzpicture}
        \end{figure}
        \begin{columns}
        \column{0.5\textwidth}
        \begin{verbatim}
    S = S "+" S
    S = "(" S ")"
    S = "int"
        \end{verbatim}
        \column{0.5\textwidth}
        \begin{figure}
        \tikzstyle{Token} = [rectangle, draw=black, text centered, fill=orange, text width=1.0cm, minimum height=0.7cm]
        \begin{tikzpicture}[node distance=0.2cm, auto, thick]
            \node (a1) [Token] {S};
            \node (a2) [Token, below=of a1, opacity=0, onslide=<3->{opacity=1}] {int};

            \draw<3-> [->] (a1) -- (a2);
        \end{tikzpicture}
        \end{figure}
        \end{columns}
    \end{frame}

    \begin{frame}[fragile]
        \frametitle{Grammar - Lookahead}
        \begin{verbatim}
            S = S "+" S
            S = "(" S ")"
            S = "int"
        \end{verbatim}
        \pause
        \begin{verbatim}
            S = E ("+" S)?
            E = "(" S ")" | "int"
        \end{verbatim}
        \pause
        \begin{verbatim}
            S = E ("+" E)*
            E = "(" S ")" | "int"
        \end{verbatim}
    \end{frame}

    \begin{frame}[fragile]
        \frametitle{Grammar - Lookahead}
        \begin{figure}
        \tikzstyle{Token} = [rectangle, draw=black, text centered, fill=orange, text width=0.3cm, minimum height=0.7cm]
        \tikzstyle{TokenInt} = [rectangle, draw=black, text centered, fill=orange, text width=1.3cm, minimum height=0.7cm]
        \tikzstyle{High} = [fill=red]
        \tikzstyle{Remove} = [opacity=0]
        \begin{tikzpicture}[node distance=0.1cm, auto, thick]
            \node (a1) [TokenInt, onslide=<2->{High}, onslide=<5->{Remove}] {int};
            \node (a2) [Token, onslide=<6->{High}, onslide=<7->{Remove}, right=of a1] {$+$};
            \node (a3) [Token, right=of a2] {$($};
            \node (a4) [TokenInt, right=of a3] {int};
            \node (a5) [Token, right=of a4] {$)$};
            \node (a6) [Token, right=of a5] {$+$};
            \node (a7) [TokenInt, right=of a6] {int};
        \end{tikzpicture}
        \end{figure}
        \begin{columns}
        \column{0.5\textwidth}
        \begin{verbatim}
    S = E ("+" S)?
    E = "(" S ")" | "int"
        \end{verbatim}
        \column{0.5\textwidth}
        \begin{figure}
        \tikzstyle{Token} = [rectangle, draw=black, text centered, fill=orange, text width=1.0cm, minimum height=0.7cm, opacity=0]
        \tikzstyle{Show} = [opacity=1]
        \begin{tikzpicture}[node distance=0.2cm, auto, thick]
            \node (a1) [Token, Show] {S};
            \node (a2) [Token, onslide=<3->{Show}, below=of a1] {E + S};
            \node (a3) [Token, onslide=<4->{Show}, below left=0.2cm and 0.1cm of a2.south west, anchor=north] {int};
            \node (a4) [Token, onslide=<7->{Show}, below right=0.2cm and 0.1cm of a2.south east, anchor=north] {E + S};
            \node (a5) [Token, onslide=<7->{Show}, below left=0.2cm and 0.1cm of a4.south west, anchor=north] {$($ S $)$};
            \node (a6) [Token, onslide=<7->{Show}, below right=0.2cm and 0.1cm of a4.south east, anchor=north] {E + S};
            \node (a7) [Token, onslide=<7->{Show}, below=of a5] {int};
            \node (a8) [Token, onslide=<7->{Show}, below=of a6] {int};

            \draw<3-> [->] (a1) -- (a2);
            \draw<4-> [->] (a2) -- (a3);
            \draw<7-> [->] (a2) -- (a4);
            \draw<7-> [->] (a4) -- (a5);
            \draw<7-> [->] (a4) -- (a6);
            \draw<7-> [->] (a5) -- (a7);
            \draw<7-> [->] (a6) -- (a8);

        \end{tikzpicture}
        \end{figure}
        \end{columns}
    \end{frame}


    \begin{frame}[fragile]
        \frametitle{CPU Grammar}
        \begin{columns}
        \column{0.5\textwidth}
        \begin{lstlisting}[]
let a = 0;
let b = 1;
let counter = 0;
let fib = 10;
while(counter < fib) {
    let tmp = b;
    b = a + b;
    a = tmp;
    counter = counter + 1;
}
print "fibonacci: " + b;
        \end{lstlisting}
        \column{0.5\textwidth}
        \pause
        \begin{verbatim}
    program = stmt* "EOF"
    stmt = let | 
        while | 
        if | 
        print | 
        exprstmt |
        scope
        \end{verbatim}
        \end{columns}
    \end{frame}

    \begin{frame}[fragile]
        \frametitle{CPU Grammar}
        \begin{columns}
        \column{0.5\textwidth}
        \begin{lstlisting}[]
let a = 0;
let b;
        \end{lstlisting}
        \column{0.5\textwidth}
        \pause
        \begin{verbatim}
    let = 
        "LET"
        "IDENT"
        ("=" expr)?
        ";"
        \end{verbatim}
        \end{columns}
    \end{frame}

    \begin{frame}[fragile]
        \frametitle{CPU Grammar}
        \begin{columns}
        \column{0.5\textwidth}
        \begin{lstlisting}[]
while(a < b) {
    let x = 1;
}
        \end{lstlisting}
        \column{0.5\textwidth}
        \pause
        \begin{verbatim}
    while = 
        "WHILE"
        "(" expr ")"
        scope
    
    scope = 
        "{"
        stmt*
        "}"
        \end{verbatim}
        \end{columns}
    \end{frame}

    \begin{frame}[fragile]
        \frametitle{CPU Grammar}
        \begin{columns}
        \column{0.5\textwidth}
        \begin{lstlisting}[]
print "test: " + a;
print c;
print "test"
        \end{lstlisting}
        \column{0.5\textwidth}
        \pause
        \begin{verbatim}
    print = 
        "PRINT"
        expr
        ";"
        \end{verbatim}
        \end{columns}
    \end{frame}

    \begin{frame}[fragile]
        \frametitle{CPU Grammar}
        \begin{columns}
        \column{0.5\textwidth}
        \begin{lstlisting}[]
-a + 1;
a = 2;
b * (c + d);
        \end{lstlisting}
        \column{0.5\textwidth}
        \pause
        \begin{verbatim}
    exprstmt = 
        expr
        ";"
    
    expr = 
        unary (op expr)?
    unary = 
        "-" unary |
        primary
    primary =
        "NUMBER" |
        "IDENT" | 
        "STRING"
        "(" expr ")"
    op = "+" | "-" | ...
        \end{verbatim}
        \end{columns}
    \end{frame}

    \begin{frame}[fragile]
        \frametitle{Operator Precedence}
        \begin{columns}
        \column{0.5\textwidth}
        \begin{lstlisting}[]
1 * 2 + 3
        \end{lstlisting}
        \only<2->{\begin{figure}
        \tikzstyle{Token} = [rectangle, draw=black, text centered, fill=orange, text width=1.0cm, minimum height=0.7cm]
        \begin{tikzpicture}[node distance=0.2cm, auto, thick]
            \node (a1) [Token] {$*$};
            \node (a2) [Token, below left=0.2cm and 0.1cm of a1.south west, anchor=north] {1};
            \node (a3) [Token, below right=0.2cm and 0.1cm of a1.south east, anchor=north] {$+$};
            \node (a4) [Token, below left=0.2cm and 0.1cm of a3.south west, anchor=north] {2};
            \node (a5) [Token, below right=0.2cm and 0.1cm of a3.south east, anchor=north] {3};

            \draw[->] (a1) -- (a2);
            \draw[->] (a1) -- (a3);
            \draw[->] (a3) -- (a4);
            \draw[->] (a3) -- (a5);

        \end{tikzpicture}
        \end{figure}}
        \column{0.5\textwidth}
        \begin{verbatim}
    exprstmt = 
        expr
        ";"
    
    expr = 
        unary (op expr)?
    unary = 
        "-" unary |
        primary
    primary =
        "NUMBER" |
        "IDENT" | 
        "STRING"
        "(" expr ")"
    op = "+" | "-" | ...
        \end{verbatim}
        \end{columns}
    \end{frame}

    \begin{frame}[fragile]
        \frametitle{Operator Precedence}
        \begin{columns}
        \column{0.5\textwidth}
        \begin{verbatim}
    exprstmt = 
        expr
        ";"
    
    expr = 
        unary (op expr)?
    unary = 
        "-" unary |
        primary
    primary =
        "NUMBER" |
        "IDENT" | 
        "STRING"
        "(" expr ")"
    op = "+" | "-" | ...
                \end{verbatim}
        \column{0.5\textwidth}
        \pause
        \begin{verbatim}
    expr = 
        mul ("*" expr)?
    mul = 
        unary ("*" mul)?
    unary = 
        "-" unary |
        primary
    primary =
        "NUMBER" |
        "IDENT" | 
        "STRING"
        "(" expr ")"
        \end{verbatim}
        \end{columns}
    \end{frame}

    \begin{frame}[fragile]
        \frametitle{Operator Precedence}
        \begin{columns}
        \column{0.4\textwidth}
        \begin{verbatim}
    *
    /
    %
    -
    +
    <
    >
    <=
    >=
    ==
    !=
    =
        \end{verbatim}
        \column{0.6\textwidth}
        \pause
        \begin{verbatim}
expr = eq
eq = 
    comp (("==" | "!=") eq)?
comp = 
    add (("<" | ">") comp)?
add = 
    mul (("+" | "-") add)?
mul = 
    unary (("*" | "/") mul)?
unary = 
    "-" unary |
    primary
primary =
    "NUMBER" | "STRING" |
    "IDENT" | 
    "(" expr ")"
        \end{verbatim}
        \end{columns}
    \end{frame}

    \begin{frame}[fragile]
    \frametitle{Why?}
    \begin{verbatim}
        A = B C
        A = B | C
        A = B*
        A = B?
        A = "TEST"
    \end{verbatim}
    \end{frame}

    \begin{frame}[fragile]
    \frametitle{Why?}
    \begin{columns}
    \column{0.1\textwidth}
    \begin{verbatim}
    A = B C
    \end{verbatim}
    \column{0.9\textwidth}
    \pause
    \begin{lstlisting}[language=Python]
def parseA(self):
    b = self.parseB()
    c = self.parseC()
    return A(b, c)
        \end{lstlisting}
    \end{columns}
    \end{frame}

    \begin{frame}[fragile]
    \frametitle{Why?}
    \begin{columns}
    \column{0.1\textwidth}
    \begin{verbatim}
    A = B | C
    \end{verbatim}
    \column{0.9\textwidth}
    \pause
    \begin{lstlisting}[language=Python]
def parseA(self):
    tok = self.peek().kind
    if(kind == Kind.B):
        return self.parseB()
    elif(kind == Kind.C):
        return self.parseC()
    raise Error
        \end{lstlisting}
    \end{columns}
    \end{frame}

    \begin{frame}[fragile]
    \frametitle{Why?}
    \begin{columns}
    \column{0.1\textwidth}
    \begin{verbatim}
    A = B*
    \end{verbatim}
    \column{0.9\textwidth}
    \pause
    \begin{lstlisting}[language=Python]
def parseA(self):
    while(self.peek().kind = Kind.B):
        self.parseB()
    return A()
        \end{lstlisting}
    \end{columns}
    \end{frame}

    \begin{frame}[fragile]
    \frametitle{Why?}
    \begin{columns}
    \column{0.1\textwidth}
    \begin{verbatim}
    A = B?
    \end{verbatim}
    \column{0.9\textwidth}
    \pause
    \begin{lstlisting}[language=Python]
def parseA(self):
    if(self.peek().kind = Kind.B):
        self.parseB()
    return A()
        \end{lstlisting}
    \end{columns}
    \end{frame}

    \begin{frame}[fragile]
    \frametitle{Why?}
    \begin{columns}
    \column{0.1\textwidth}
    \begin{verbatim}
    A = "TEST"
    \end{verbatim}
    \column{0.9\textwidth}
    \pause
    \begin{lstlisting}[language=Python]
def parseA(self):
    self.consume(Kind.TEST)
    return A()
        \end{lstlisting}
    \end{columns}
    \end{frame}

    \begin{frame}[fragile]
    \begin{lstlisting}[language=Python]
def parseEq(self):
    expr = self.parseComp()

    op = self.test(Kind.CMPEQ)
    if(op):
        right = self.parseEq()
        expr = Binary(op, expr, right)

    return expr
    \end{lstlisting}
    \end{frame}

    \begin{frame}[fragile]
    \begin{lstlisting}[language=Python]
def parseLet(self):
    self.consume(Kind.LET)
    ident = self.consume(Kind.IDENT)
    expr = None

    if(self.test(Kind.EQUALS)):
        expr = self.parseExpression()

    self.consume(Kind.SEMICOLON)
    return Declaration(ident, expr)
    \end{lstlisting}
    \end{frame}

    \begin{frame}[fragile]
    \frametitle{Interpreter}
    \begin{lstlisting}[]
1 * (-2 + 3)
    \end{lstlisting}
    \begin{figure}
    \tikzstyle{Token} = [rectangle, draw=black, text centered, fill=orange, text width=1.0cm, minimum height=0.7cm]
    \begin{tikzpicture}[node distance=0.2cm, auto, thick]
        \node (a1) [Token] {B($*$)};
        \node (a2) [Token, below left=0.2cm and 0.1cm of a1.south west, anchor=north] {L(1)};
        \node (a3) [Token, below right=0.2cm and 0.1cm of a1.south east, anchor=north] {B($+$)};
        \node (a4) [Token, below left=0.2cm and 0.1cm of a3.south west, anchor=north] {U($-$)};
        \node (a5) [Token, below right=0.2cm and 0.1cm of a3.south east, anchor=north] {L(3)};
        \node (a6) [Token, below=of a4] {L(2)};

        \draw[->] (a1) -- (a2);
        \draw[->] (a1) -- (a3);
        \draw[->] (a3) -- (a4);
        \draw[->] (a3) -- (a5);
        \draw[->] (a4) -- (a6);
    \end{tikzpicture}
    \end{figure}
    \end{frame}

    \begin{frame}[fragile]
    \frametitle{Interpreter}
    \begin{columns}
    \column{0.5\textwidth}
    \begin{figure}
    \tikzstyle{Token} = [rectangle, draw=black, text centered, fill=orange, text width=1.0cm, minimum height=0.7cm]
    \begin{tikzpicture}[node distance=0.2cm, auto, thick]
        \node (a1) [Token] {B($*$)};
        \node (a2) [Token, below left=0.2cm and 0.1cm of a1.south west, anchor=north] {L(1)};
        \node (a3) [Token, below right=0.2cm and 0.1cm of a1.south east, anchor=north] {B($+$)};
        \node (a4) [Token, below left=0.2cm and 0.1cm of a3.south west, anchor=north] {U($-$)};
        \node (a5) [Token, below right=0.2cm and 0.1cm of a3.south east, anchor=north] {L(3)};
        \node (a6) [Token, below=of a4] {L(2)};

        \draw[->] (a1) -- (a2);
        \draw[->] (a1) -- (a3);
        \draw[->] (a3) -- (a4);
        \draw[->] (a3) -- (a5);
        \draw[->] (a4) -- (a6);
    \end{tikzpicture}
    \end{figure}
    \column{0.5\textwidth}
    \pause
    \begin{verbatim}
eval(B(*)) = 
    eval(L(1)) * 
    eval(B(+))
eval(L(1)) = 1
eval(B(+)) = 
    eval(U(-)) + 
    eval(L(3))
eval(U(-)) = - eval(L(2))
eval(L(2)) = 2
eval(L(3)) = 3
    \end{verbatim}
    \end{columns}
    \end{frame}

    \begin{frame}[fragile]
    \frametitle{Environment}
    \begin{columns}
    \column{0.5\textwidth}
    \begin{lstlisting}[]
let x = 1;
let y = 2;
    \end{lstlisting}
    \column{0.5\textwidth}
    \begin{figure}
    \tikzstyle{Token} = [rectangle, draw=black, text centered, fill=orange, text width=4.0cm, minimum height=0.7cm]
    \begin{tikzpicture}[node distance=0.2cm, auto, thick]
        \node (a1) [Token] {x = 1 \\ y = 2};
    \end{tikzpicture}
    \end{figure}
    \end{columns}
    \end{frame}

    \begin{frame}[fragile]
    \frametitle{Environment}
    \begin{columns}
    \column{0.5\textwidth}
    \begin{lstlisting}[]
let x = 1;
let y = 2;
{
    let z = 3;
}
print z;
    \end{lstlisting}
    \column{0.5\textwidth}
    \begin{figure}
    \tikzstyle{Token} = [rectangle, draw=black, text centered, fill=orange, text width=4.0cm, minimum height=0.7cm]
    \begin{tikzpicture}[node distance=0.3cm, auto, thick]
        \node<1-> (a1) [Token] {x = 1 \\ y = 2};
        \node<2-> (a2) [Token, below=of a1] {z = 3};
        \draw<2-> [->] (a2) -- (a1);
    \end{tikzpicture}
    \end{figure}
    \end{columns}
    \end{frame}
    
\end{document}